%%There is just one section and two subsections.
\documentclass[11pt]{article}
%\documentclass[aps]{article}
\usepackage{amsmath}
\usepackage{amssymb}
\usepackage{verbatim}
%\usepackage{aas_macros}
\usepackage{graphicx}
\usepackage{float}
\usepackage{afterpage}
\newcommand{\del}{\nabla}
\newcommand{\boldnabla}{\boldmath $\nabla$ \unboldmath}
\usepackage{lscape}
\usepackage{cite}
\textwidth = 6.5 in
\textheight = 9 in
\oddsidemargin = 0.0 in
\evensidemargin = 0.0 in
\topmargin = 0.0 in
\headheight = 0.0 in
\headsep = 0.0 in
%\parskip = 0.2in
\parindent = 0.2in
\newcommand{\dblspace}{\renewcommand{\baselinestretch}{1.8}\normalsize}
\newcommand{\snglspace}{\renewcommand{\baselinestretch}{1}\normalsize}
\newcommand{\qe}{\stackrel{?}{=}}
\newcommand{\av}[1]{\left \langle #1 \right \rangle}


\begin{document}
\section{Inner Product Notes}
IP1 is the equation
\begin{equation}
\frac{1}{2} \oint {^2}R (D^k v_i)(D_k v_j) {\rm d \Omega} = \frac{2}{3} 4 \pi \delta_{ij}.
\end{equation}
IP2 is the equation
\begin{equation}
\frac{1}{2} \oint {^2}R (D^k v_i)(D_k v_j) {\rm d A} = \frac{2}{3}  \pi A \delta_{ij}.
\end{equation}
%IP3 is the equation
%\begin{equation}
%\frac{1}{2} \oint {^2}R (D^k v_i)(D_k v_j) {\rm d A} = \frac{2}{3}  \pi A s \delta_{ij}.
%\end{equation}
IP4 is the equation
\begin{equation}
\frac{1}{2} \oint  (D^k v_i)(D_k v_j) {\rm d \Omega} = \frac{2}{3} 4 \pi \delta_{ij}.
\end{equation}
IP5 is the equation
\begin{equation}
\frac{1}{2} \oint  (D^k v_i)(D_k v_j) {\rm d A} = \frac{2}{3}  \pi A \delta_{ij}.
\end{equation}
%IP6 is the equation
%\begin{equation}
%\frac{1}{2} \oint  (D^k v_i)(D_k v_j) {\rm d A} = \frac{2}{3}  \pi A s \delta_{ij}.
%\end{equation}
(Note that ${\rm dA} = r^2 \psi^4 {\rm d\Omega}$.) In general,
\begin{eqnarray}
{^2}R &=& \frac{2}{r^2 \psi^4} (1-2{~_s}\nabla^2 \ln \psi)\\
S_{ij} &=& \left( \begin{matrix} 1&0 \\ 0 & \sin^2 \theta \end{matrix} \right) \\
h_{ij} &=& r^2 \psi^4 S_{ij}.
\end{eqnarray}

\noindent
For a general surface, the LHS of [IP1, IP2] become
\begin{eqnarray}
\frac{1}{2} \oint {^2}R (D^k v_i)(D_k v_j) {~\rm d [\Omega, A]}
&=& \frac{1}{2} \oint \frac{2}{r^2 \psi^4}(1-2{~_s}\nabla^2 \ln \psi) (h^{kl} \partial_l v_i) (\partial_k v_j) {~\rm d [\Omega,A]} \\
&=& \oint \frac{1}{r^2 \psi^4}(1-2{~_s}\nabla^2 \ln \psi) \frac{1}{r^2 \psi^4} S^{kl} (\partial_l v_{i})(\partial_k v_{j}) {~\rm d [\Omega,A]} \\
&=& \oint \frac{1}{r^4 \psi^8}(1-2{~_s}\nabla^2 \ln \psi) S^{kl} (\partial_l v_{i})(\partial_k v_{j}) {~\rm d [\Omega,A]}
\end{eqnarray}

\noindent
In SpherePack, a real scalar function $f(\theta, \phi)$ is represented as
\begin{equation}
f(\theta, \phi) = \sum_{l=0}^{\infty} {\sum_{m=0}^{l}}' P_l^m(\cos \theta)\{ a_{lm} \cos(m\phi) - b_{lm}\sin(m\phi)\},
\end{equation}
where $\sum'$ indicates that the $m=0$ term is to be multiplied by $1/2$.  As a shorthand, define
\begin{eqnarray}
a_{lm}^{f} &\equiv& P_l^m(\cos \theta) a_{lm} \cos(m \phi) \\
b_{lm}^{f} &\equiv& P_l^m(\cos \theta) b_{lm} \sin(m \phi) 
\end{eqnarray}
for a given expansion of the function $f$.

\noindent
Define the following notation for expansion using only the $l=0$ terms:
\begin{eqnarray}
\av{r^2 \psi^4} &\equiv& \sum_{l=0}^{0}{\sum_{m=0}^{l}}' P_l^m(\cos \theta)\{ a_{lm} \cos(m\phi) - b_{lm}\sin(m\phi)\} \\
&=& \frac{1}{2} a_{00}^{r^2 \psi^4}
\end{eqnarray}

\begin{comment}
\noindent
(is this a good idea?) As further shorthand, define
\begin{equation}
Y_{1j} \equiv P_1^j (\cos \theta)\{ a_{1j} \cos(j\phi) - b_{1j}\sin(j\phi)\}
\end{equation}
\end{comment}


\section{Sphere}
For a sphere, the $l=1$ terms of $L$ are non-zero, and the $l=0$ terms of $r^2 \psi^4$ are non-zero.  We see that
\begin{eqnarray}
r^2 \psi^4 &=& \sum_{l=0}^{0} {\sum_{m=0}^{0}}' \{ a_{lm}^{r^2 \psi^4}  - b_{lm}^{r^2 \psi^4} \} = \frac{1}{2} a_{00}^{r^2 \psi^4} \\
L_i &=& \sum_{l=1}^{1} {\sum_{m=i}}' \{ a_{lm}^{L}  - b_{lm}^{L} \} = a_{1i}^{L} - b_{1i}^{L}
\end{eqnarray}
Then $v$ can be expressed as
\begin{eqnarray}
v_i &\equiv& \sum_{l=0}^{\infty} {\sum_{m=0}^{l}}' \{ a_{lm}^{v}  - b_{lm}^{v} \} \\
{_s}\nabla^2 v_i &=& -\sum_{l=0}^{\infty} {\sum_{m=0}^{l}}' l(l+1)\{a_{lm}^{v}  - b_{lm}^{v} \} \\
{_s}\nabla^2 v_i &=& -2 r^2 \psi^4 L_i = - a_{00}^{r^2 \psi^4} \{ a_{1i}^{L} - b_{1i}^{L} \}\\
\implies v_i &=& -\frac{1}{2} a_{00}^{r^2 \psi^4} \{ a_{1i}^{L}-b_{1i}^{L} \}.
\end{eqnarray}


\subsection{Unit Sphere}
For a unit sphere,
\begin{eqnarray}
r &=& 1 \\
\psi &=& 1 \\
v_i &=& -(1) L_{i} =   -\{ a_{1i}^{L}-b_{1i}^{L} \} .
\end{eqnarray}

\noindent
The relations IP1 and IP2 become
\begin{eqnarray}
\oint \frac{1}{r^4 \psi^8}(1-2{_s}\nabla^2 \ln \psi) S^{kl} (\partial_l v_{i})(\partial_k v_{j}) {~\rm d [\Omega,A]}
&=& \oint S^{kl} (\partial_l L_i)(\partial_k L_j) {~\rm d [\Omega,A]}
\end{eqnarray}



\subsection{Arbitrary Sphere}
For an arbitrary sphere,
\begin{eqnarray}
r &=& {\rm constant} \\
\psi &=& {\rm constant} \\
v_i &=& -r^2 \psi^4 L_i = -\frac{1}{2} a_{00}^{r^2 \psi^4} \{ a_{1i}^{L}-b_{1i}^{L} \}.
\end{eqnarray}

\noindent
The relations IP1 and IP2 become
\begin{eqnarray} \nonumber
&&\oint \frac{1}{r^4 \psi^8}(1-2{_s}\nabla^2 \ln \psi) S^{kl} (\partial_l v_{i})(\partial_k v_{j}) {~\rm d [\Omega,A]} \\
&=& \oint \frac{1}{r^4 \psi^8} S^{kl} (\partial_l r^2 \psi^4 L_{i})(\partial_k r^2 \psi^4 L_{j}) {~\rm d [\Omega,A]} \\
&=& \oint S^{kl} (\partial_l  L_{i})(\partial_k  L_{j}) {~\rm d [\Omega,A]}
\end{eqnarray}


\section{Arbitrary Surface}
\subsection{With Ricci term}
For an arbitrary surface, $l \ne 1$ terms may contribute to the expansions of $L$ and $r^2 \psi^4$.  The function $L$ is represented as
\begin{eqnarray}
L_i  &=& \sum_{l=0}^{\infty}{\sum_{m=0}^l}' P_l^m \{ a_{lm} \cos(m\phi) - b_{lm}\sin(m\phi)\} \\
&=& \{ a_{1i}^{L}-b_{1i}^{L}  \} + \frac{1}{2} a_{00}^{L} + \sum_{l=2}^{\infty}{\sum_{m=0}^l}' \{ a_{lm}^{L}-b_{lm}^{L}  \}.
\end{eqnarray}
Similarly, for the conformal factor,
\begin{equation}
r^2 \psi^4 = \frac{1}{2} a_{00}^{r^2 \psi^4} + \sum_{l=1}^{\infty}{\sum_{m=0}^l}' \{ a_{lm}^{r^2 \psi^4}-b_{lm}^{r^2 \psi^4}  \} 
\end{equation}


\noindent
Then the $v_i$ terms become
\begin{eqnarray}
v_i &=& -r^2 \psi^4 L_i \\ \nonumber
&=& -\frac{1}{2} a_{00}^{r^2 \psi^4} \{ a_{1i}^{L}-b_{1i}^{L} \} \\ \nonumber
&&-\frac{1}{2}a_{00}^{r^2\psi^4} \left[ \frac{1}{2}a_{00}^{L}+\sum_{l=2}^{\infty}{\sum_{m=0}^l}' \{ a_{lm}^{L}-b_{lm}^{L}\} \right] \\
&&-\frac{1}{2}a_{00}^{L}\sum_{l=1}^{\infty}{\sum_{m=0}^l}' \{ a_{lm}^{r^2 \psi^4}-b_{lm}^{r^2 \psi^4}\} \\ \nonumber
&& -\{ a_{1i}^{L}-b_{1i}^{L} \} \sum_{l=1}^{\infty}{\sum_{m=0}^l}' \{ a_{lm}^{r^2 \psi^4}-b_{lm}^{r^2 \psi^4}\} \\ \nonumber
&& -\sum_{l=1}^{\infty}{\sum_{m=0}^l}' \{ a_{lm}^{r^2 \psi^4}-b_{lm}^{r^2 \psi^4}\} 
        \sum_{l=2}^{\infty}{\sum_{m=0}^l}' \{ a_{lm}^{L}-b_{lm}^{L}\}
\end{eqnarray}

\newpage
\noindent
Then [IP1,IP2] become
\begin{eqnarray}
&&\frac{1}{2} \oint \frac{2}{r^2\psi^4} \left(1-2~_s\nabla^2 \ln \psi \right) \frac{1}{r^2 \psi^4} S^{kl} (\partial_l v_i)(\partial_k v_j) {\rm d [\Omega,A]} \\
&=& \oint \frac{1}{r^4 \psi^8}  S^{kl} (\partial_l v_i)(\partial_k v_j) {\rm d [\Omega,A]} -  \oint \frac{1}{r^4 \psi^8}  \left(2~_s\nabla^2 \ln \psi \right)S^{kl} (\partial_l v_i)(\partial_k v_j) {\rm d [\Omega,A]} \\
&=& \oint \frac{1}{r^4 \psi^8}  S^{kl} (r^4\psi^8) (\partial_l L_i)(\partial_k L_j) {\rm d [\Omega,A]} + \oint \frac{1}{r^4 \psi^8}  S^{kl} L_i L_j(\partial_l r^2\psi^4)(\partial_k r^2\psi^4) {\rm d [\Omega,A]} \\ \nonumber
&&-  \oint \frac{1}{r^4 \psi^8}  \left(2~_s\nabla^2 \ln \psi \right)S^{kl} (\partial_l v_i)(\partial_k v_j) {\rm d [\Omega,A]} \\
\end{eqnarray}
Does $\partial L_{nm}$ contribute to the integral for $n \ne 1$?  Is it correct that we are only looking for $X_{00}, \partial X_{1,m}, \nabla^2 X_{2,m} \ldots$ terms? If so, then:
\begin{eqnarray}
&=& \oint S^{kl} (\partial_l L_{1i})(\partial_k L_{1j}) {\rm d [\Omega,A]} 
+ \oint \frac{1}{r^4 \psi^8}  S^{kl} L_{00} L_{00} \left[ \partial_l (r^2\psi^4)_{1i} \right] \left[ \partial_k (r^2\psi^4)_{1j} \right] {\rm d [\Omega,A]} \\ \nonumber
&&- \oint  \left(2~_s\nabla^2 \ln \psi \right)S^{kl} (\partial_l L_i)(\partial_k L_j) {\rm d [\Omega,A]} \\ \nonumber
&&- \oint \frac{1}{r^4 \psi^8} \left(2~_s\nabla^2 \ln \psi \right) S^{kl} L_{00} L_{00}\left[ \partial_l (r^2\psi^4)_{1i} \right] \left[ \partial_k (r^2\psi^4)_{1j} \right] {\rm d [\Omega,A]}
\end{eqnarray}
If the above assumption is correct, then we can/must simplify the $\left(2~_s\nabla^2 \ln \psi \right)$ term.  How do we associate the $m$-terms of $\psi_{2m}$ with the $i$-terms of $v_i$?
\begin{eqnarray}
&=& \oint S^{kl} (\partial_l L_{1i})(\partial_k L_{1j}) {\rm d [\Omega,A]} 
+ \oint a_{00}^{1/r^4 \psi^8}  S^{kl} a_{00}^{L} a_{00}^{L} \left[ \partial_l (r^2\psi^4)_{1i} \right] \left[ \partial_k (r^2\psi^4)_{1j} \right] {\rm d [\Omega,A]} \\ \nonumber
&&- \oint  \left(2~_s\nabla^2 \ln \psi_{2m} \right)S^{kl} (\partial_l L_i)(\partial_k L_j) {\rm d [\Omega,A]} \\ \nonumber
&&- \oint a_{00}^{1/r^4 \psi^8} \left(2~_s\nabla^2 \ln \psi \right) S^{kl}a_{00}^{L} a_{00}^{L} \left[ \partial_l (r^2\psi^4)_{1i} \right] \left[ \partial_k (r^2\psi^4)_{1j} \right] {\rm d [\Omega,A]}
\end{eqnarray}
If my assumptions are correct, this is the result.  The first term looks like the result for the sphere.

\subsection{Without Ricci term}
For [IP4,IP5] we get:
\begin{eqnarray}
&&\frac{1}{2} \oint \frac{1}{r^2 \psi^4} S^{kl} (\partial_l v_i)(\partial_k v_j) {\rm d [\Omega,A]} \\
&=& \frac{1}{2} \oint \frac{1}{r^2 \psi^4} S^{kl} r^4\psi^8 (\partial_l L_i)(\partial_k L_j) {\rm d [\Omega,A]} \\ \nonumber
&& + \frac{1}{2} \oint \frac{1}{r^2 \psi^4} S^{kl} L_{00} L_{00} \left[ \partial_l (r^2\psi^4)_{1i} \right] \left[ \partial_k (r^2\psi^4)_{1j} \right] {\rm d [\Omega,A]} \\ 
&=& \frac{1}{2} \oint a_{00}^{r^2\psi^4} S^{kl}  (\partial_l L_i)(\partial_k L_j) {\rm d [\Omega,A]} \\ \nonumber
&&+\frac{1}{2} \oint a_{00}^{1/r^2 \psi^4} S^{kl} a_{00}^L a_{00}^L  \left[ \partial_l (r^2\psi^4)_{1i} \right] \left[ \partial_k (r^2\psi^4)_{1j} \right] {\rm d [\Omega,A]}
\end{eqnarray}
The ${\rm d\Omega}$ integral of the first term looks like $1/2$ IP2 for a sphere.











%\bibliographystyle{apsrev}
%\bibliography{bibfile}


\end{document}
