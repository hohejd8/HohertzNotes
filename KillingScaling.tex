%%There is just one section and two subsections.
\documentclass[11pt]{article}
%\documentclass[aps]{article}
\usepackage{amsmath}
\usepackage{amssymb}
\usepackage{verbatim}
%\usepackage{aas_macros}
\usepackage{graphicx}
\usepackage{float}
\usepackage{afterpage}
\newcommand{\del}{\nabla}
\newcommand{\boldnabla}{\boldmath $\nabla$ \unboldmath}
\usepackage{lscape}
\usepackage{cite}
\textwidth = 6.5 in
\textheight = 9 in
\oddsidemargin = 0.0 in
\evensidemargin = 0.0 in
\topmargin = 0.0 in
\headheight = 0.0 in
\headsep = 0.0 in
%\parskip = 0.2in
\parindent = 0.2in
\newcommand{\dblspace}{\renewcommand{\baselinestretch}{1.8}\normalsize}
\newcommand{\snglspace}{\renewcommand{\baselinestretch}{1}\normalsize}
\newcommand{\qe}{\stackrel{?}{=}}
\newcommand{\av}[1]{\left \langle #1 \right \rangle}


\begin{document}
\section{Killing Scaling}
Let the Killing vector be proportional to some basis vector such that $\xi = s \widetilde{A}$, where $s$ is specific to the fact that the $l=1$ modes of $L$ are normalized.  Then
\begin{eqnarray}
\frac{3}{16 \pi} \oint ~^2R \widetilde{A} \cdot \widetilde{A} {\rm d \Omega} &=& \frac{1}{s} \\
\frac{3}{16 \pi} \oint ~^2R (s \widetilde{A}) \cdot (s \widetilde{A}) {\rm d \Omega} &=& 
\frac{3}{16 \pi} \oint ~^2R \xi \cdot \xi {\rm d \Omega} = s 
\end{eqnarray}
Similarly, let $\zeta$ be proportional to $\widetilde{A}$ such that $\zeta = \alpha \widetilde{A}$, where $\alpha$ represents the fact that the $l=1$ modes are not necessarily normalized.  Then
\begin{equation}
\frac{3}{16 \pi} \oint ~^2R (\alpha \widetilde{A}) \cdot (\alpha \widetilde{A}) {\rm d \Omega} = 
\frac{3}{16 \pi} \oint ~^2R \zeta \cdot \zeta {\rm d \Omega} = \frac{\alpha^2}{s} \equiv \beta
\end{equation}

\noindent
How do we relate $\xi$ to $\zeta$ directly using the spherical harmonic functions?  We are only interested in the $Y_a^m$ coefficients, so look at the $Y_1^m$ terms:
\begin{eqnarray}
Y_1^0 &=& \sqrt{\frac{3}{4 \pi}} \cos \theta \\
Y_1^{\pm 1} &=& \mp \sqrt{\frac{3}{8 \pi}} \sin \theta e^{\pm i \phi}.
\end{eqnarray}
We can write the orthogonal directions $(x,y,z)$ in terms of trig functions on a sphere:
\begin{alignat}{3}
n_1 &= \sin \theta \cos \phi &= {x} \\
n_2 &= \sin \theta \sin \phi &= {y}\\
n_3 &= \cos \theta &= {z}
\end{alignat}
Then in terms of the $Y_1^m$'s, we have
\begin{equation}
\begin{split}
\label{n-Y}
n_1 &= -\frac{1}{2} \sqrt{\frac{8 \pi}{3}} \left( Y_1^1 - Y_1^{-1} \right) \\
n_2 &= -\frac{1}{2i} \sqrt{\frac{8 \pi}{3}} \left( Y_1^1 + Y_1^{-1} \right) \\
n_3 &= \sqrt{4 \pi}{3}  Y_1^0
\end{split}
\end{equation}

\noindent
Note: $n_i$ are real, which implies
\begin{eqnarray}
n_1 &=& -\frac{1}{2} \sqrt{\frac{8 \pi}{3}} \left( Y_1^1 - Y_1^{-1} \right) \\
&=&  -\frac{1}{2} \sqrt{\frac{8 \pi}{3}} \sin \theta \left( -e^{i \phi} - e^{-i\phi}  \right) \\
&=& -\frac{1}{2} \sqrt{\frac{8 \pi}{3}} \sin \theta \left( -Y_1^{-1*} + Y_1^{1*} \right) \\
n_1 &=& n_{1*}
\end{eqnarray}
and similarly for $n_2, n_3$.

\noindent
Imagine a scenario in which we rotate the coordinates such that the primary axis $z'$ differs from the original axis $z$ by an amount $(\gamma, \delta)$. [Inserting a graphic might be handy here.] In terms of the old coordinates $(\theta, \phi)$, we can write the expansion in the new coordinates $(\theta', \phi')$ as (see Merzbacher Eq.~16.60):
\begin{equation}
Y_1^0 (\theta', \phi') = \sqrt{\frac{4 \pi}{3}} \sum_m Y_1^m(\theta, \phi) Y_1^m(\gamma, \delta)
\end{equation}
Can we make a similar statement about $n_3$?
\begin{eqnarray}
n_3(\theta', \phi') &=& \sqrt{\frac{4 \pi}{3}} Y_1^0 (\theta', \phi') \\
&=& \sqrt{\frac{4 \pi}{3}} \sqrt{\frac{4 \pi}{3}} \sum_m Y_1^m(\theta, \phi) Y_1^m(\gamma, \delta) \\
&=& \frac{4 \pi}{3} \left( \frac{3}{8 \pi} \sin \theta e^{i \phi} \sin \gamma e^{-i \delta} 
+ \frac{3}{4 \pi} \cos \theta \cos \gamma 
+ \frac{3}{8 \pi} \sin \theta e^{-i \phi} \sin \gamma e^{i \delta} \right) \\
&=& \cos \theta \cos \gamma
+ \frac{4 \pi}{3} \frac{3}{8 \pi} \sin \theta \sin \gamma \left( e^{i \phi} e^{-i \delta} + e^{-i \phi}e^{i \delta} + 0 \right) \\
&=& n_3(\theta, \phi) n_3(\gamma, \delta) \\ \nonumber
&&+ \frac{2 \pi}{3} \frac{3}{8 \pi} \sin \theta \sin \gamma \left( 2e^{i \phi} e^{-i \delta} + 2e^{-i \phi}e^{i \delta} 
+ e^{i \phi}e^{i \delta} - e^{i \phi}e^{i \delta} + e^{-i \phi}e^{-i \delta} - e^{-i \phi}e^{-i \delta}  \right) \\
&=&  n_3(\theta, \phi) n_3(\gamma, \delta) \\ \nonumber
&&+ \frac{2 \pi}{3} \frac{3}{8 \pi} \sin \theta \sin \gamma 
\left( e^{i \phi}e^{i \delta} + e^{i \phi}e^{-i \delta} + e^{-i \phi}e^{i \delta} + e^{-i \phi}e^{-i \delta} \right) \\ \nonumber
&&+ \frac{2 \pi}{3} \frac{3}{8 \pi} \sin \theta \sin \gamma 
\left( -e^{i \phi}e^{i \delta} + e^{i \phi}e^{-i \delta} + e^{-i \phi}e^{i \delta} - e^{-i \phi}e^{-i \delta} \right) \\
&=& n_3(\theta, \phi) n_3(\gamma, \delta) \\ \nonumber
&& + \frac{2 \pi}{3} \left( -Y_1^{1}(\theta, \phi)Y_1^{-1*}(\gamma, \delta) + Y_1^{1}(\theta, \phi)Y_1^{1*}(\gamma, \delta) + Y_1^{-1}(\theta, \phi)Y_1^{-1*}(\gamma, \delta) - Y_1^{-1}(\theta, \phi)Y_1^{1*}(\gamma, \delta) \right) \\ \nonumber
&& + \frac{2 \pi}{3} \left( Y_1^{1}(\theta, \phi)Y_1^{-1*}(\gamma, \delta) + Y_1^{1}(\theta, \phi)Y_1^{1*}(\gamma, \delta) + Y_1^{-1}(\theta, \phi)Y_1^{-1*}(\gamma, \delta) + Y_1^{-1}(\theta, \phi)Y_1^{1*}(\gamma, \delta) \right) \\
&=&  n_3(\theta, \phi) n_3(\gamma, \delta) \\ \nonumber
&& + \frac{2 \pi}{3} \left( Y_1^{1}(\theta, \phi) - Y_1^{-1}(\theta, \phi) \right) \left( Y_1^{1*}(\gamma, \delta) - Y_1^{-1*}(\gamma, \delta) \right) \\ \nonumber
&& + \frac{2 \pi}{3} \left( Y_1^{1}(\theta, \phi) + Y_1^{-1}(\theta, \phi) \right) \left( Y_1^{1*}(\gamma, \delta) + Y_1^{-1*}(\gamma, \delta) \right) \\ \nonumber
&=&  n_3(\theta, \phi) n_3(\gamma, \delta)  + n_1(\theta, \phi) n_{1*}(\gamma, \delta) + n_2(\theta, \phi) n_{2*}(\gamma, \delta)\\ 
n_3(\theta', \phi') &=& \sum_m n_m(\theta, \phi)n_{m*}(\gamma, \delta) =  \sum_m n_m(\theta, \phi)n_{m}(\gamma, \delta) 
\end{eqnarray}

\noindent
When $\gamma, \delta$ are chosen carefully such that the $z'$ axis is now a symmetry axis, then the $m \ne 0$ modes vanish in the $Y_1^m(\theta', \phi')$ expansion.

\newpage
\noindent
Let $\vec{A}$ be an arbitrary vector which has a spherical harmonic decomposition
\begin{equation}
\vec{A} = a_1^{0} Y_1^{0} + a_1^{1} Y_1^{1} + a_1^{-1} Y_1^{-1}.
\end{equation}
When we take advantage of the orthogonality of the 
(non-normalized)
$n_i$ from Eq.~\ref{n-Y}, we can write the dot product of $\vec{A}$ with itself  as
\begin{comment}
\begin{eqnarray}
\vec{A} \cdot \vec{A} &=& 
\left[ \frac{1}{4}\frac{8 \pi}{3} (a_1^{1} Y_1^{1} - a_1^{1} Y_1^{1}) (a_1^{1} Y_1^{1} - a_1^{1} Y_1^{1}) \right]
+\left[ \frac{-1}{4}\frac{8 \pi}{3} (a_1^{1} Y_1^{1} + a_1^{1} Y_1^{1}) (a_1^{1} Y_1^{1} + a_1^{1} Y_1^{1}) \right] \\ \nonumber
&&+ \left[ \frac{4 \pi}{3} (a_1^{0})(a_1^{0}) \right] \\
&=& \frac{2 \pi}{3} \left[ (a_1^{1})^2 + 2 a_1^{1} a_1^{-1}  + (a_1^{-1})^2 
- (a_1^{1})^2 + 2 a_1^{1} a_1^{-1} - (a_1^{-1})^2 \right] +   \frac{4 \pi}{3} (a_1^{0})^2 \\
&=& \frac{8 \pi}{3} a_1^{1} a_1^{-1} +  \frac{4 \pi}{3} (a_1^{0})^2 
\end{eqnarray}
\end{comment}
\begin{eqnarray}
\vec{A} \cdot \vec{A} &=& 
\left[  (a_1^{1} Y_1^{1} + a_1^{-1} Y_1^{-1}) (a_1^{1} Y_1^{1} + a_1^{-1} Y_1^{-1}) \right]
+ i^2 \left[  (a_1^{1} Y_1^{1} - a_1^{-1} Y_1^{-1}) (a_1^{1} Y_1^{1} - a_1^{-1} Y_1^{-1}) \right] \\ \nonumber
&&+ \left[  (a_1^{0}Y_1^{0})(a_1^{0}Y_1^{0}) \right] \\
&=& \left[ (a_1^{1}Y_1^{1})^2 + 2 a_1^{1} a_1^{-1}Y_1^{1}Y_1^{-1}  + (a_1^{-1}Y_1^{-1})^2 
- (a_1^{1}Y_1^{1})^2 + 2 a_1^{1} a_1^{-1}Y_1^{1}Y_1^{-1} - (a_1^{-1}Y_1^{-1})^2 \right]  \\ \nonumber
&&+    (a_1^{0}Y_1^{0})^2 \\
&=& 4 a_1^{1} a_1^{-1} +   (a_1^{0}Y_1^{0})^2 \\
&=&    (a_1^{0}Y_1^{0})^2 
\end{eqnarray}
The cross-terms vanish because the spherical harmonics are orthogonal in $l$ and $m$.


From Eq.~\ref{n-Y}, we can say that
\begin{eqnarray}
b_1 &=& (a_1^{1}-a_1^{1}) \\
b_2 &=& (a_1^{1}+a_1^{1}) \\
b_3 &=& a_1^{0}.
\end{eqnarray}














%\bibliographystyle{apsrev}
%\bibliography{bibfile}


\end{document}
